\documentclass{report}

\pagestyle{empty}

\setlength{\topmargin}{-0.8in}
\setlength{\textheight}{9.7in}
\setlength{\textwidth}{6.9in}
\setlength{\oddsidemargin}{-0.2in}
\setlength{\evensidemargin}{-0.5in}

\newcommand{\itemvs}{1ex}
\newcommand{\sectitlevs}{1.2ex}
\newcommand{\secendvs}{1.5ex}

\newcommand{\indentone}{0.5in}
\newcommand{\widthone}{6.4in}
\newcommand{\pboxone}{\hspace*{\indentone}\parbox[t]{\widthone}}


\newcommand{\indenttwo}{0.75in}
\newcommand{\widthtwo}{6.15in}
\newcommand{\pboxtwo}{\hspace*{\indenttwo}\parbox[t]{\widthtwo}}

\raggedright

\begin{document}
%
%%%%%%%%%%%%%%%%%%%%%%%%%%%%%%%%%%%%%%%%%%%%%%%%%%%%%%
\textbf{\LARGE Dr Roscoe A. Bartlett, Ph.D.} \\
\textbf{Address:} 4317 Buckeystown Pike, Apt. E, Frederick, MD 21704; \textbf{Phone:} 240-600-4172
\\
{}\textbf{Email:} {}\texttt{rabartl@sandia.gov};
{}\textbf{Website:} {}\texttt{https://bartlettroscoe.github.io}
%
%
\framebox[\textwidth]{}\\[2.0ex]
%
%
%%%%%%%%%%%%%%%%%%%%%%%%%%%%%%%%%%%%%%%%%%%%%%%%%%%%%%
\textbf{\Large Experience}\\[\sectitlevs]
%
{}\pboxone{\textbf{Sandia National Laboratories (SNL)} (2016-present)\\
Multiphysics Applications group, computational science and engineering, software development, software engineering processes and practices, DOE Q clearance} \\[\itemvs]
%
{}\pboxtwo{\textbf{Lead for Advanced Technology Deployment Mitigation (ATDM) Development Operations (DevOps):}  Develop processes and tools for development and integration workflows for advanced numerical software on next generation platforms.  Lead effort to harden and stablize Trilinos for ATDM next-generation platforms and ATDM application customers.} \\[\itemvs]
%
{}\pboxtwo{\textbf{Interoperable Design for Extreme-scale Application Software (IDEAS) for the Exascale Project:} Creating/refining development workflows for Computational Science \& Engineering (CSE) projects. SE training and outreach for the CSE community. CSE software lifecycles and sustainability.  (See \texttt{https://ideas-productivity.org})}
\\[\itemvs]
%
%
{}\pboxone{\textbf{Oak Ridge National Laboratory (ORNL)} (2011-2016)\\
Computational Engineering \& Energy Sciences group, computational science and engineering, software development, software engineering processes and practices, DOE Q clearance} \\[\itemvs]
%
{}\pboxtwo{\textbf{Software engineering lead for DOE Nuclear Energy Hub CASL:} Develop processes, specialize tools, conduct education and training in Lean and Agile software engineering for the Consortium for the Advanced Simulation of Light-water reactors (CASL).  Lead infrastructure team creating and maintaining software integration process involving commercial, national lab, and university software to produce CASL Virtual Environment for Reactor Applications (VERA) and performing releases of VERA.  Lead developer of the Tribal Build Integrate and Test System (TriBITS) used for CASL VERA software and other software (e.g. Trilinos).  Led early software architecture for VERA enabling technologies as well as design of mathematical simulation coupling and analysis toolkit.  Helped revise CASL milestone-driven process to be more consistent with Lean/Agile, lead initial creation of milestone project management site, and helped to establish milestone protocols and controls.} \\[\itemvs]
%
{}\pboxtwo{\textbf{Interoperable Design for Extreme-scale Application Software (IDEAS):} Led Lean/Agile project management creation. Creating/refining development workflows for Computational Science \& Engineering (CSE) projects. SE training and outreach for the CSE community. CSE software lifecycles and sustainability.  Standardization of build systems for CSE software.  (See \texttt{https://ideas-productivity.org})}
\\[\itemvs]
%
{}\pboxtwo{\textbf{Software engineering lead for Trilinos project:} Continuing to oversee and manage software engineering foundations and processes.}
\\[\itemvs]
%
{}\pboxone{\textbf{Sandia National Laboratories (SNL)} (2001-2011)\\
Optimization and Uncertainty Estimation Department, computational science and engineering, software development, software engineering, DOE Q clearance} \\[\itemvs]
%
{}\pboxtwo{\textbf{R\&D of numerical algorithms:} Performed research and implemented software for new novel algorithms for gradient-based numerical optimization of mathematical models (Ph.D. background).} \\[\itemvs]
%
{}\pboxtwo{\textbf{Algorithm and software development:} Developed many software packages as well as generic utility software (over 6000 commits to Trilinos alone, almost double any other Trilinos developer).} \\[\itemvs]
%
{}\pboxtwo{\textbf{Software engineering lead for Trilinos project:} Oversaw and managed the software engineering foundations and infrastructure for a larger computational science and engineering project (\texttt{trilinos.sandia.gov}).} \\[\itemvs]
%
{}\pboxtwo{\textbf{Object-oriented software and C++ consulting:} Expert in object-oriented design and expert in C++ design and programming; used as a center-wide resource in OO and C++ (see {}\texttt{https://bartlettroscoe.github.io//readingList.html}).} \\[\itemvs]
%
{}\pboxtwo{\textbf{Software engineering training leadership:} Led and coordinated training and mentorship of critical software skills including arranging formal multi-day classes and book reading groups.} \\[\itemvs]
%
{}\pboxtwo{\textbf{Computer languages/software:} C++ (guru), C, Python, Perl, CMake/CTest, Fortran77, Windows IDE, Linux/Unix, Emacs, MPI, and others} \\[\itemvs]
%
{}\pboxtwo{\textbf{Project leadership:} Led several projects including the 2007 Vertical Integration Milestone effort, the 2008 and 2009 SIERRA Trilinos Integration teams, and the 2009 and 2010 NEAMS WF Infrastructure sub-team.} \\[\itemvs]
%
{}\pboxtwo{\textbf{Public speaking:} Gave numerous technical presentations at conferences and other venues.} \\[\sectitlevs]
%
{}\pboxone{\textbf{University of Maryland Baltimore County (UMBC)} (1995-1996)\\ Research and teaching assistant for Dr. Govind Rao}\\[\sectitlevs]
%
{}\pboxone{\textbf{Bartlett For Congress}, Frederick MD (1995-1996)\\
Information Systems Manager/Developer.  Developed a campaign management package called \textit{Campaign Pro} (accounting, FEC reporting, fundraising support etc.) used by incumbent U.S. congressional campaign.}\\[\itemvs]
%
{}\pboxtwo{\textbf{Computer languages/software}: Relational database design, Microsoft Access (SQL, Access Basic, Data Access Objects (DAO))} \\[\sectitlevs]
%
%%%%%%%%%%%%%%%%%%%%%%%%%%%%%%%%%%%%%%%%%%%%%%%%%%%%%%
\textbf{\Large Education}\\[\sectitlevs]
%
{}\pboxone{\textbf{Carnegie Mellon University (CMU)}, Pittsburgh PA (1996-2001)\\ Ph.D. in Chemical Engineering (August 2001).  Thesis Title {}\textit{Object-Oriented Methods for Successive Quadratic Programming for Large-Scale Process Optimization}}\\[\itemvs]
%
{}\pboxtwo{\textbf{GPA}: 3.76/4.0, Cum Laude} \\[\itemvs]
%
{}\pboxtwo{\textbf{Nonlinear programming}: Successive Quadratic Programming, Quadratic Programming, Linear Solvers.  Theoretical analysis and practical algorithms.} \\[\itemvs]
%
{}\pboxtwo{\textbf{Object-oriented modeling and design}: Unified Modeling Language (UML), Design Patterns etc.}\\[\itemvs]
%
{}\pboxtwo{\textbf{Computer languages/software}: C++ (ANSI/ISO Standard, Standard Library(STL)), Fortran 77, Perl, Matlab, Windows, Unix/Linux, Latex}\\[\itemvs]
%
{}\pboxone{\textbf{University of Maryland Baltimore County (UMBC)} (1993-1995) \\ B.S. Chemical Engineering} \\[\itemvs]
%
{}\pboxtwo{\textbf{GPA}: 4.0/4.0, Summa Cum Laude} \\[\itemvs]
%
{}\pboxtwo{\textbf{Honors}: Graduated first in class, Outstanding Graduating Chemical Engineer} \\[\itemvs]
%
{}\pboxone{\textbf{Frederick Community College}, Frederick MD
(1991-1993)}\\
%
{}\pboxtwo{\textbf{GPA}: 3.955/4.0.  {}\textbf{Honors}: Sigma Xi Award for Science and Engineering} \\[\secendvs]
%
{}\pboxone{\textbf{Hood College}, Frederick MD (1991-1992)} \\
%
{}\pboxtwo{\textbf{GPA}: 4.0/4.0. {}\textbf{Classes}: Cell Biology, Genetics} \\[\secendvs]
%
%\pagebreak
%
%%%%%%%%%%%%%%%%%%%%%%%%%%%%%%%%%%%%%%%%%%%%%%%%%%%%%%
{}\textbf{\Large Professional Awards}
%
\begin{enumerate}
%
{}\item ONRL Director's Team award for Research Accomplishment for AP100 LWR work for CASL, 2014
%
{}\item Nominee for Consortium for the Advanced Simulation of Light-water reactors (CASL) Knight Award, 2013
%
{}\item ORNL Significant Event Team Award for the completion and delivery of version 2.0 of the Virtual Environment for Reactor Applications (VERA), 2012
%
{}\item ORNL Computing and Computational Science Directorate's Distinguished Employee Award for Leading the RSICC 2012 VERA Release, October 2012
%
{}\item Sandia Award for Excellence for dramatically enhancing cross-organizational collaboration through tighter integration of SIERRA and Trilinos, 2010
%
{}\item Sandia Award for Excellence for SIERRA Trilinos Integration infrastructure, 2008
%
{}\item Sandia Employee Recognition Award for ASC Xyce/Charon/Algorithms Integration Team, 2007
%
{}\item Sandia Award for Excellence for expertise and leadership for Vertical Integration Milestone, 2007
%
{}\item Sandia Award for Excellence for release of MOOCHO optimization software in Trilinos 7.0, 2007
%
{}\item Sandia Certificate of Appreciation for development and release of Trilinos 7.0 solver framework, 2007 
%
{}\item Sandia Employee Recognition Award Nomination for numerical software and interfaces, 2006 
%
{}\item Sandia Employee Recognition Award for Trilinos Project Team, 2005 
%
{}\item Sandia Employee Recognition Award Nomination for Trilinos Development Team, 2004 
%
{}\item Sandia Award for Excellence for water security modeling and optimization LDRD, 2004 
%
{}\item Sandia Award for Excellence for source inversion of chem-bio releases, 2004 
%
{}\item Sandia Employee Recognition Award Nomination for computational algorithms for water homeland security team, 2004
%
{}\item Sandia Certificate of Appreciation for software engineering advancements in Trilinos, 2004 
%
{}\item SC2004 HPC Software Challenge Award, 2004
%
{}\item R\&D 100 Award for Trilinos 3.1, 2004 
%
{}\item Sandia Employee Recognition Award for DAKOTA Optimization
Team, 2002
%
\end{enumerate}
%
{\tiny .}\\[\secendvs]
%
%\pagebreak
%
%%%%%%%%%%%%%%%%%%%%%%%%%%%%%%%%%%%%%%%%%%%%%%%%%%%%%%
{}\textbf{\Large Selected Publications} \\[\sectitlevs]
%
\pboxone{\texttt{https://bartlettroscoe.github.io/\#\_Publications}}
%
\begin{enumerate}
%
{}\item Bartlett, Roscoe.  A Roadmap for Sustainable Ecosystems of CSE Software.  Accepted Short paper.  Computational Science and Engineering Software Sustainability and Productivity Challenges (CSESSP) Workshop.  October 15-16, 2015
%
{}\item Bartlett, Roscoe.  TriBITS Developers Guide and Reference. Oak Ridge National Lab. CASL-U-2014-0075-000-b.  March 2014
%
{}\item Bartlett, Roscoe, Michael Heroux, and Jim Willenbring.  Agile Lifecycles for Research-driven CSE Software. ASCR Workshop on Software Productivity for Extreme-Scale Science.  October 2013
%
{}\item Bartlett, Roscoe.  Overview of Software Challenges in CSE. ASCR Workshop on Software Productivity for Extreme-Scale Science.  October 2013
%
{}\item Bartlett, Roscoe.  Fortran Isolates the CSE Community. ASCR Workshop on Software Productivity for Extreme-Scale Science.  October 2013
%
{}\item Bartlett, Roscoe, Michael Heroux, and Jim Willenbring.  Overview of the TriBITS Lifecycle Model : A Lean/Agile Software Lifecycle Model for Research-based Computational Science and Engineering Software. To be published in proceedings of the First Workshop on Maintainable Software Practices in e-Science, part of the IEEE International Conference on eScience 2012.  October 2012
%
{}\item Bartlett, Roscoe, Michael Heroux, and Jim Willenbring.  TriBITS Lifecycle Model Version 1.0: A Lean/Agile Software Lifecycle Model for Research-based Computational Science and Engineering and Applied Mathematical Software. SAND2012-0561. Sandia National Laboratories. February 2012
%
{}\item Bartlett, Roscoe. Teuchos C++ Memory Management Classes, Idioms, and Related Topics: The Complete Reference (A Comprehensive Strategy for Safe and Efficient Memory Management in C++ for High Performance Computing). SAND2010-2234, Sandia National Laboratories. May 2010
%
{}\item Bartlett, Roscoe. Thyra Coding and Documentation Guidelines (TCDG) Version 1.0. SAND2010-2051. Sandia National Laboratories. May 2010
%
{}\item Bartlett, Roscoe. Mathematical and High-Level Overview of MOOCHO: The Multifunctional Object-Oriented arCHitecture for Optimization. SAND2009-3969, Sandia National Laboratories. June 2009
%
{}\item Bartlett, Roscoe. Integration Strategies for Computational Science \& Engineering Software.  SAND2004-3268, Second International Workshop on Software Engineering for Computational Science and Engineering, 2009
%
{}\item Bartlett, Roscoe. Teuchos::RCP Beginner's Guide (An Introduction to the Trilinos Smart Reference-Counted Pointer Class for (Almost) Automatic Dynamic Memory Management in C++).  SAND2004-3268, Sandia National Laboratories, 2007 (Updated November 2008)
%
{}\item Bartlett, Roscoe, Daniel Dunlavy, and Tim Shead. SAND2008-7593, Trilinos CMake Evaluation. Sandia National Laboratories, October 2008
%
{}\item Bartlett, Roscoe. Derivation of forward and adjoint sensitivities for ODEs and DAEs, SAND2007-6699, Sandia National Laboratories. October 2007
%
{}\item Bartlett, Roscoe. Daily Integration and Testing of the Development Versions of Applications and Trilinos: A stronger foundation for enhanced collaboration in application and algorithm research and development, SAND2007-7040, Sandia National Laboratories, October 2007
%
{}\item Bartlett, Roscoe, Scott Collis, Todd Coffey, David Day, Mike Heroux, Rob Hoekstra, Russell Hooper, Roger Pawlowski, Eric Phipps, Denis Ridzal, Andy Salinger, Heidi Thornquist, and Jim Willenbring. ASC Vertical Integration Milestone. SAND2007-5839, Sandia National Laboratories, 2007
%
{}\item Bartlett, Roscoe, Bart van Bloemen Waanders, and Martin Berggeren. Hybrid Differentiation Strategies for Simulation and Analysis of Applications in C++. ACM TOMS, Vol. 35, No. 1, Article 1,
July 2008
%
{}\item Bartlett, Roscoe. Thyra Linear Operators and Vectors: Overview of Interfaces and Support Software for the Development and Interoperability of Abstract Numerical Algorithms. SAND2007-5984, Sandia National Laboratories, 2007
%
{}\item Bartlett, Roscoe, and Lorenz Biegler. QPSchur: A dual, active-set, Schur-complement method for large-scale and structured convex quadratic programming. Optim Eng, vol 7, p. 5-32, 2006
%
{}\item Bartlett, Roscoe, Bart van Bloemen Waanders, and Michael Heroux. Vector Reduction/Transformation Operators, ACM Transactions on Mathematical Software. Vol. 30, No. 1, p. 62-85, 2004
%
\end{enumerate}
%
{\tiny .}\\[\secendvs]
%
%\pagebreak
%
%%%%%%%%%%%%%%%%%%%%%%%%%%%%%%%%%%%%%%%%%%%%%%%%%%%%%%
{}\textbf{\Large Selected Presentations} \\ [\sectitlevs]
%
\pboxone{\texttt{https://bartlettroscoe.github.io/\#\_Presentations}}
%
\begin{enumerate}
%
{}\item Bartlett, Roscoe. Some Agile Best Technical Practices for the Development of Research-Based CSE Software. SciDev Workshop. University of Illinois at Urbana–Champaign. August 18, 2015
%
{}\item Bartlett, Roscoe. Overview of Git Workflows for CSE Software.  Trilinos Spring Developers Meeting.  Albuquerque, NM.  May 13, 2015
%
{}\item Bartlett, Roscoe. TriBITS: Tribal Build, Integrate, and Tests System.  SIAM Computational Science \& Engineering Conference, Salt Lake City, Utah, March 14, 2015
%
{}\item Bartlett, Roscoe. Breaking Selected Packages out of Trilinos and Importing other Packages: Motivations, concerns, workflows, examples.  Trilinos Users Group Meeting, October 29, 2014
%
{}\item Bartlett, Roscoe. Multi-Repository Development and Integration using TriBITS.  Trilinos Users Group Meeting, October 29, 2014
%
{}\item Bartlett, Roscoe. Multi-Repository Development and Integration in CASL using TriBITS.  Trilinos Users Group Meeting, November 6, 2013
%
{}\item Bartlett, Roscoe. Trilinos Adoption of the TriBITS Lifecycle Model.  Trilinos Users Group Meeting, November 1, 2012
%
{}\item Bartlett, Roscoe. Overview of the TriBITS Lifecycle Model.  First Workshop on Maintainable Software Practices in e-Science, e-Science 2012, October 9, 2012
%
{}\item Bartlett, Roscoe. TriBITS Lifecycle Model Version 1.0.  ORNL Computer Science and Mathematics Division, Oak Ridge, TN, August 21, 2012
%
{}\item Bartlett, Roscoe. TriBITS Lifecycle Model and Agile Technical Practices for Trilinos?  Trilinos Developers Meeting 2012, Albuquerque, NM, May 22, 2012
%
{}\item Bartlett, Roscoe. The State of Trilinos Software Engineering: Recent Progress, Current Status, and Future Issues.  2010-7789C, Trilinos Users Group Meeting 2010, Albuquerque, NM, November 4, 2010
%
{}\item Bartlett, Roscoe. Trilinos Software Engineering Technologies and Integration Capability Area Overview.  2010-7704C, Trilinos Users Group Meeting 2010, Albuquerque, NM, November 2, 2010
%
{}\item Bartlett, Roscoe.  Overview Software Life-cycle and Integration Issues for CS\&E R\&D Software and Experiences from Trilinos (Part I).  SIAM Parallel Computing Conference, Seattle, February 24, 2010
%
{}\item Bartlett, Roscoe.  Overview Software Life-cycle and Integration Issues for CS\&E R\&D Software and Experiences from Trilinos (Part II, Integration Issues).  SIAM Parallel Computing Conference, Seattle, February 24, 2010
%
{}\item Bartlett, Roscoe. Trilinos Release Improvement Issues. 2009-7555P, Trilinos Users Group Meeting 2009, Albuquerque, NM, November 5, 2009
%
{}\item Bartlett, Roscoe. Trilinos Software Engineering Status and Future Issues.  2009-7704P, Trilinos Users Group Meeting 2009, Albuquerque, NM, November 5, 2009
%
{}\item Bartlett, Roscoe. Trilinos Software Engineering Technologies and Integration Capability Area Overview.  2009-7512P, Trilinos Users Group Meeting 2009, Albuquerque, NM, November 3, 2009
%
{}\item Bartlett, Roscoe. Integration Strategies for Computational Science and Engineering Software.  2009-0655 C, Second International Workshop and Software Engineering for Computational Science \& Engineering, Vancouver, Canada, May 23, 2009
%
{}\item Bartlett, Roscoe. Almost Continuous Integration for the Co-Development of Highly Integrated Applications and Third Party Libraries.  2009-1114P, Sandia Software Engineering Seminar Series, October 2008
%
{}\item Bartlett, Roscoe. Maintaining the Stability of Trilinos Dev: Stable vs. Experimental Code.  2008-7714P, Trilinos Users Group Meeting 2008, October 2008
%
{}\item Bartlett, Roscoe. APP + Trilinos Integration: Status, Opportunities, and Challenges.  2008-7716P, Trilinos Users Group Meeting 2008, October 2008
%
{}\item Bartlett, Roscoe. Trilinos Software Engineering Technologies and Integration.  2008-7718P, Trilinos Users Group Meeting 2008, October 2008
%
{}\item Bartlett, Roscoe. Teuchos Utility Classes for Safer Memory Management in C++.  2008-7717P, Trilinos Users Group Meeting 2008, October 2008
%
{}\item Bartlett, Roscoe. CMake For Trilinos Developers.  2008-7715P, Trilinos Users Group Meeting 2008, October 2008
%
{}\item Bartlett, Roscoe. CMake Trilinos?  2008-7721P, Trilinos Users Group Meeting 2008, October 2008
%
{}\item Bartlett, Roscoe. Open-Source Software for Interfacing and Support of Large-scale Embedded Nonlinear Optimization.  2008-7720C, INFORMS Annual Meeting, October 2008
%
{}\item Bartlett, Roscoe. New Teuchos Utility Classes for Safer Memory Management in C++. SAND2007-7237C, 2007 Trilinos User's Group Meeting, Sandia National Laboratories, November 2007 (Updated August 2008)
%
{}\item Bartlett, Roscoe.  ModelEvaluator: Scalable, Extensible Interface Between Embedded Nonlinear Analysis Algorithms and Applications.  High Performance Computing Software Week, Boston, April 3, 2008
%
{}\item Bartlett, Roscoe.  Stratimikos: Unified Wrapper to Trilinos Linear Solvers and Preconditioners.  High Performance Computing Software Week, Boston, April 3, 2008
%
{}\item Bartlett, Roscoe.  Overview of the Vertical Integration of Trilinos Solver Algorithms in a Production Application Code.  SIAM Parallel Computing Conference, Atlanta, March 13, 2008
%
{}\item Bartlett, Roscoe.  Teuchos::RCP: An Introduction to the Trilinos Smart Reference-Counted Pointer Class for (Almost) Automatic Dynamic Memory Management in C++.  SAND2005-4855P, Sandia National Laboratories, 2005 (Updated February 2008)
%
{}\item Bartlett, Roscoe. Embedded Sensitivities and Optimization: From Research to Applications. SAND2008-0769P, Optimization and Uncertainty Estimation Department Review, Sandia National Laboratories, January 2008 (Updated February 2008)
%
{}\item Bartlett, Roscoe. Daily Integration and Testing of the Development Versions of Applications and Trilinos: A stronger foundation for enhanced collaboration in application and algorithm research and development. SAND2007-7236C, Sandia Software Engineering Seminar Series, Sandia National Laboratories, October 2007
%
{}\item Bartlett, Roscoe. Using Thyra and Stratimikos to Build Blocked and Implicitly Composed Solver Capabilities. SAND2007-7231C, 2007 Trilinos User's Group Meeting, Sandia National Laboratories, November 2007
%
{}\item Bartlett, Roscoe. Using FY07 ASC Vertical Integration Milestone: Overview, Lessons Learned, and Next Steps. SAND2007-7401C, 2007 Trilinos User's Group Meeting, Sandia National Laboratories, November 2007
%
\end{enumerate}
%
\end{document}
